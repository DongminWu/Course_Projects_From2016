\documentclass[article]{aaltoseries}
\usepackage[utf8]{inputenc}


\begin{document}
 
%=========================================================

\title{Deep learning algorithms on mobile devices}

\author{Dongmin Wu
\\\textnormal{\texttt{dongmin.wu@aalto.fi}}} % Your Aalto e-mail address

\affiliation{\textbf{Tutor}: Jukka K. Nurminen} % First and last name of your tutor

\maketitle

%==========================================================

\begin{abstract}

  Since the Deep learning technology was pulished, this technology is
  implemented in various field and obtained excited result. The Deep learning technology can acquire 
  dependable abstraction from various data, which has benefit on different application, 
  including movement detection, natural language processing and image recognition. 
  Although some of the Deep learning technology 
  already reduced the complex of neural network to a acceptable size for web service, the 
 scale of the Deep learning technology is still large. According to the high dependency on computation resources,
 The mobile computing
  devices such as mobile phone, wearable devices and tablet is can rarely directly executing the Deep learning technology. 
  This paper addresses and analysis current problems
   of applying the Deep learning algorithm on mobile devices and list some existing solutions
  . Through the comparison among different solutions, we are able to 
  plot a clear view about how to applying Deep learning algorithms on mobile devices.

\vspace{3mm}
\noindent KEYWORDS: Deep learning, mobile devices, 

\end{abstract}


%============================================================


\section{Introduction}


Deep learning has been widely used in past few years. As the main algorithm of Deep learning technology,
 the neural network algorithm 
also has considerable increase and became the state-of-the art for solving different pattern recognition
problems.

Neural network is inspired by the neural system in human brain. A neural network will 
try to model pattern of data by using large number of intelligent nodes, which are named perceptron. 
After gathering those computational units together, due to the high complexity of neural network, 
This algorithm has the potential to fetch non-linear features from the training data 
and build a more precise pattern.

As the result, in the image, voice and complex pattern recognition field, the performance of Deep learning
generally better then other Machine Learning algorithms.  

However, as the complexity is one of the characteristics in Deep learning technology, some obviously
issues came up with the applying of that technology. Generally, Deep learning algorithms cost most 
on it training process. For example, when deployed on a mobile device, a typical CNN (AlexNet \cite{NIPS2012_4824}) 
spends approximately 2.6 minutes in a forward process for one image. However, the amount of computation is 
still large
during the predicting procedure. Currently, Deep learning algorithms mainly run on the GPUs or special processors, 
the reason is those processors have better capabilities in parallel computation.

In addition to that, the energy consumption of Deep learning algorithms are significant too. [example]

Deep learning also have considerable contributions on mobile devices, for example mobile phones, 
wearable devices and vehicle electronics. Mobile devices have rather intensive resources than 
servers and large computers. On the other hand, mobile devices can hardly afford the energy consumption
of normal Deep learning algorithms. Under this situation, some approaches showed up.

In previous study\cite{Ota:2017}, researchers comprehensively listed different Deep learning algorithms,
reviewed some software frameworks and hardware platforms for executing neural network on mobile devices.
At last they presented some applications running on mobile devices integrated with Deep learning technology. 
In addition to that, we still need a more detailed researching on the reason of preventing the deploying of 
high performance Deep learning technologies.

This paper first analysis the characteristics of Deep learning and neural network algorithms,
listing the different among current algorithms. Then, we will address some issue of applying a high 
performance Deep learning technologies on mobile devices.
At last this paper will show some approaches of solving those problems. Besides, this paper will try to analysis the 
advantages and disadvantages of each approaches.

[structure of paper should combine with last Paragraph]



% 2.1 Problem we faced

% 2.2 how to solve this problem in this paper? --> By comparing different solutions

% 3. Explaination the structure of this paper


%============================================================


\section{Background}

In this section, a brief introduction of Deep learning and mobile devices will be made. Those fundamental 
knowledge helps us to understand the context of this paper better.


%------------------------------------------------------------


\subsection{Deep Learning and neural network}

The neural network in Deep learning can be called Deep Neural Network(DNN), which is one of the most promising 
part of current machine learning methods. 

\subsubsection{History of Deep Neural Network}

% 1. history  1page
Back to middle 1900's there is already a paper introduced a Artificial Neural Network (ANN)\cite{Warren1943}, authors
of that paper raise up the first shallow ANN which mimics the neural network system of human brain. That ANN are not
able to learn models from data, but following researches extended the capabilities of ANN and generated unsupervised 
learning algorithm and supervised algorithm subsequently.

In the 1970s and 1980s, back-propagation learning algorithm was found. Because of efficiency the application of back-propagation
reached its peak in the middle of 80's. LeCun applied this algorithm on the convolutional neural network 
for the first time on 1989, which has significant impact to the Deep learning area. After that, the Cresceptron Model was introduced
and the using of max-pooling layers in the neural network architecture was widely used in modern Deep learning technology.

After the year of 2010, Deep learning has another wave of popularity because there are more affordable GPUs with powerful 
parallel computation capacities. AlexNet\cite{NIPS2012_4824} earned the first prize of 
the 2012 ImageNet Large-Scale Visual Recognition Challenge (ILSVRC). 

From then on, the researching of DNN was considerably accelerated. Google, Facebook, Apple and Amazon has beed published
their own papers in this area. In 2015, Google introduced their GoogLeNet\cite{GoogLeNet}, whose Inception model composing the convolutional 
neural network in a new way that no sequentially arranged layers are possible. In the same year, the ResNet\cite{ResNet} of Microsoft won 
the ILSVRC with the error rate of 3.6\%, which is empirically smaller than the error rate of human beings.




\subsubsection{Characteristic of Deep Neural Network}
% 2. characteristic 1page

The Deep Neural Network higher hierarchy than ANNs, which means a large amount of hidden layers\cite{MAL-006}. 
That difference makes DNN can understand more complex model than ANN, which means the Deep learning or especially DNN
has better behavior on describing non-linear objects, including image, voice, text and bioinformatics.

Besides, comparing to Machine learning technology, deep learning are more suitable for various tasks. The basic concept
of DNN is multi-layers neural network, that algorithm will produce pattern by itself only rely on a large data set
, on the contrary side of some algorithms specifically designed for tasks. With DNN technology, users can tackle different
issue with one implementation of DNN. 

As the widely using of Big data technology, the old machine learning technologies like Support Vector Machine (SVM) are
not such suitable for processing a large amount of data. On the other hand, because of the discovering of back-propagation
algorithm, the efficiency of DNN is relatively higher than traditional Machine learning algorithms.

DNN is more flexible as well. Since the DNN is consisted by multiple neutrons, the amount of neutrons can be adjusted
according to the requirement of different task. This characteristic made DNN has large potential of applying. So far, 
DNN have already impacted our life, services like Google translate, AlphaGo and Siri are the examples of successful applications.






%------------------------------------------------------------


\subsection{Current mobile devices} %2 pages

In this paper, the mobile devices are considered as a group of electronic devices with
computational units that has good mobility and provides the interactive features to human beings.
As the range of mobile is quite large, in this paper, we specifically discuss 3 representative 
mobile devices: Mobile phone, wearable devices and vehicle devices.




%history fo mobile devices
\subsubsection{History of mobile devices}

As the most common mobile devices in our daily life, mobile phone is one of the most familiar 
electronics to normal people. For example, there are nearly 300,000 mobile phones have been 
sold during the second quarter of 2016 \cite{moblePhoneSale}. In 1973 the first handheld mobile phone
was showed by John F. Mitchiell and Martin Copper in America. Since then the mobile phone 
changed the modern life. In 2007, Apple Inc. published the first multi-touch smartphone,
 which made the mobile mobile computation become true.
  Currently, almost all the mobile phones are in this form. 


Wearable devices, like Nike+, Google Glass and Apple watch are produced in the last decade. They presents
a type of on-body electronics in addition to mobile phones. They mainly have specific features, some of them
are designed for monitoring body status while others have limited interactive function.

The last category of mobile devices discussing in this paper is the vehicle devices. From the oldest vacuum tube
car radios to recent automatic driving systems, electronics in vehicles acted an important role while people
driving. Along with the development of mobile electronic technologies, the vehicle devices are becoming more convenient
and computational.



\subsubsection{Characteristic of mobile devices}

Because of the limited resources and specific usage, mobile devices in general do not have a high performance 
universal computation core. Alternatively, they have some specifically designed computation units like DSP processor,
baseband processor and coprocessors.

In mobile phones the power of processors considerably increased in past few years. But the limitation still existed, 
one of the main reason is the energy consumption. Mobile phones has limited size and need meet the requirement continually
using, that limitation makes the computation capability of mobile phones is relatively lower than desktop computers.

As for other wearable devices, they have more limited battery capacity and slower speed. In addition to that, 
they generally have more real-time requirement than mobile phones. Most of them are required to be able to monitoring
movement immediately.

Vehicle devices have the most computation resources comparing to the other two. There are currently some self-driving
car like Waymo exists, most of them are driven by Deep learning technology. But safety requirement of vehicle devices
is highest among others too.  



%------------------------------------------------------------





\section{Challenges of applying Deep learning on mobile devices}

This section provides examples of more complex things.


%------------------------------------------------------------


\subsection{computational requirement}
Pruning
Model compression

\subsection{Energy cost}

\subsection{Balance of accuracy and resources consumption}





%------------------------------------------------------------


\section{Current approaches}

\subsection{algorithm level}

\subsection{software level}

\subsection{hardware level}

\subsection{comparison}
%------------------------------------------------------------




%============================================================


\section{Conclusion}

To be added.


%============================================================


\bibliographystyle{plain}
\bibliography{cs-seminar}

\end{document}
